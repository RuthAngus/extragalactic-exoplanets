% Motivation
A key question in the field of exoplanets today is how whether different
stellar populations host different planet populations.
Hundreds of extra-Solar planets have been discovered in the Milky Way, but an
extra-Galactic exoplanet has never been found since their host stars are so
faint and transit signals are relatively small, providing a signal-to-noise
challenge.
The Dark Energy Camera may provide the first real opportunity to
search for extragalactic exoplanets.
The properties of exoplanets outside our own galaxy may provide important
clues for star and planet formation.
Although some exoplanet surveys have targeted planet populations outside the
Solar neighborhood (for example the galactic bulge and the globular cluster 47
Tuc), no survey has successfully detected extrasolar planets in a stellar
population as dramatically different from the thin disk as those Magellanic
clouds.
% Goal
We propose to target the Magellanic clouds, both large and small to both 1)
detect the first extra-galactic exoplanet and 2) infer differences in the
population of hot Jupiters in the Clouds, contrasted with the hot Jupiter
population of the Milky Way.
% Method & data
We will observe the Large and Small Magellanic clouds with high cadence over a
series of nights in g-band with the Dark Energy Camera.
% Results
Our simulations demonstrate that our observing strategy is optimal for a
search for hot Jupiters.
We expect to discover a few hundred hot Jupiters in this survey, providing a
statistical sample of exoplanets, perfect for a population analysis.
% Interpretation
The first detection of even a single extragalactic planet would be a
significant milestone for the field of exoplanets and a huge step forward in
exoplanet research.
This discovery would lead to new insights into the efficacy of planet
formation outside the Milky Way.
