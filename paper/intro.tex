% Motivation
\section{Introduction}

One of the key questions in astrophysics today is whether different stellar
populations host different planet populations.
With the K2, and TESS and WFIRST surveys it may be possible to detect
exoplanets outside the thin disk of the galaxy and explore the population of
planets in the galactic bulge, thick disk and possibly even the Halo of the
Milky Way.
We propose to search for extra-Solar planets outside the Milky Way {\it
entirely}: in the Large and Small Magellanic clouds.
% This will be one of the first surveys with the potential to discover
% extragalactic exoplanets and could reveal characteristics of the populations
% of exoplanets in other galaxies.
% Ultimately this survey could reveal Universal planet formation pathways and
% how host galaxies influence the habitability of their planets.
% Hundreds of extra-Solar planets have been discovered in the Milky Way, but an
% extragalactic exoplanet has never been found due to the challenge of searching
% for 1\% flux variations of such distant stars.
Although some exoplanet surveys have targeted planet populations outside the
Solar neighborhood (for example the galactic bulge and the globular cluster 47
Tuc), no survey has successfully detected extrasolar planets in a stellar
population as dramatically different from the thin disk of the Milky Way as
those Magellanic clouds.
Given its ability to perform exquisite photometry, the Dark Energy Camera
(DECam) may provide an opportunity to search for extragalactic exoplanets.
% Goal
We propose to target the Magellanic clouds, both Large and Small, in a high
cadence photometric survey, aiming to detect the first extragalactic exoplanet
and to provide constraints on the population of giant planets, brown dwarfs
and small stars in dwarf galaxy environments.
% Method & data
% Results
In order to assess the feasibility of this study, we simulated light curves
with realistic white noise properties and cadence.
These simulations indicate that we may be able to detect inflated hot Jupiters
orbiting Sun like stars.
% Interpretation
The first detection of even a single extragalactic planet would be a
significant milestone for the field of exoplanets and a huge step forward in
exoplanet research.
With the detection of one or more giant planets and a planet detection
pipeline with well understood completeness properties, it may be possible to
place constraints on the {\it occurrence rates} of hot Jupiters in the Large
and Small Magellanic clouds and thus, characterize the efficacy of planet
formation in these radically different stellar populations.
