% METHOD
%=============================================================================
% Light curve simulations
%   - Intro
%   - Planet injections
%   - Planet detection
%   - Optimizing cadence and observing strategy
% Results
%   - Prospects: yields and predictions
% Discussion & Conclusion
%   - Caveats and challenges
%   - Implications
%=============================================================================

% Light curve simulations
%   - Intro
We tested the potential for finding extra-galactic planets in this survey by
simulating light curves with realistic white noise properties, cadence and
intermittent time coverage from ground-based observations.
We also used these simulations to optimize our observing strategy.

%   - Planet injections
We injected exoplanet transits into our simulated light curves

%   - Planet detection

%   - Optimizing cadence and observing strategy
We tested a range of cadences by injecting planets into light curves with
different cadences and corresponding white noise level and running BLS on
these light curves.
We tested cadences of 3, 5, 8 and 10 minutes with SNRs of 7.0\%, 5.3\%, 4.2\%
and 3.8\% respectively.
We found that the most rapid cadence: 3 minutes of integration time resulted
in the largest number of successfully detected exoplanets, despite decreased
photometric precision per exposure.

% Results
%   - Prospects: yields and predictions

% Discussion & Conclusion
%   - Caveats and challenges
\begin{itemize}
\item{Correlated noise}
\item{Stellar variability}
\item{Impact parameter -- all transits were generated with an impact
parameter of zero, meaning that they transit across the center of the star and
have the longest duration (and deepest) possible transit.}
\end{itemize}

%   - Implications

